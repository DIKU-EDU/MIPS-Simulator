This report describes the development of a MIPS simulator, intended to support
the operating system KUDOS. The simulator will be written in C, and will
support the most important processor features and I/O devices, required to run
KUDOS, such as the translation lookaside
buffer (TLB), memory management unit (MMU), user and kernel CPU modes,
multiple cores (SMP), and I/O device emulation. \\


\subsection{Motivation}
KUDOS is a small operating system skeleton intended to be used by students
attending operating system project courses at university of Copenhagen.
It is used to explore operating system concepts by extending and improving on
existing system.
Initially, KUDOS targets the MIPS32 architecture, which leverages on the
advantages of a
reduced instruction set computing --- RISC.\\
To ease the development and debugging of KUDOS, it is desireable to run the
operating system
in a simulated machine. This enables the students and other developers to
better inspect the state of the machine, as well as making up
for the difference in the hardware of the host machine.

\subsection{A simulator}
A simulator is a program or a machine, that models some key characteristics
and functions of a given target.  The purpose of a
simulator is to be able to look inside the simulation and inspect the properties
and behaviours, that would otherwise only be seen in the real target.
A possible by-product of a simulator is that the simulation model will emulate
the target and its behaviour, practically immitating the target.\\
An emulator, on the other side, is intended to only mimick the behaviour of
a target on the outside, not correctly reflecting the internal state of the target.
Emulators are often used as a substitute for the real targets.
The difference between emulation and simulation is therefore, that unlike simulating a system,
an emulator only imitates the outward behaviour of its target, and it is hard
to predict how the target would act internally.\\
For example, mobile developers often use an emulator to test their
applications.
Instead of having thousands of real smartphones, they can simply emulate\footnote{
To be precise, they are emulating the hardware devices, but the applications they
are testing, are being simulated.}
the devices on their computers, saving both time and money.\\
A common use of a simulator is in the aircraft business, where flight-simulators
are being used for both pilot training, engineering, design, and many other
purposes. While it gives a very good insight on how a simulated airplane might
react in given situations, it does not actually move the users from one point
to another.\\
The purpose of our project is to write a program used at the Computer Systems
course at Copenhagen University, that can emulate KUDOS, while at the same
time, exposes the internal workings of the machine, for the students and
other participants to inspect and study.\\
Our target is therefore to simulate the academically important and relevant
parts, while others will be emulated.

