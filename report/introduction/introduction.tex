This report describes the development of a MIPS simulator, intended to support
the operating system KUDOS. The simulator will be written in C, and will
support the most important processor features such as the translation
lookaside
buffer (TLB), memory management unit (MMU), user and kernel CPU modes,
multiple
cores (SMP), and I/O device emulation. \\


\subsection{Motivation}
KUDOS is a small operating system skeleton intended to be used by students
attending operating system project courses at university of Copenhagen.
It is used to explore operating system concepts by extending and improving on
existing system.
Initially, KUDOS targetsthe MIPS architecture, which leverages on the
advantages of a
reduced instruction set computing - RISC.\\
To ease the development and debugging of KUDOS, it is desireable to run the OS
in a simulated machine. This enables the students and other developers to
better inspect the state of the machine while executing, as well as making up
for the difference in the hardware of the host machine.

\subsection{A simulator}
Simulation is the act of imitating the operation of an existing system. In our
case, we will be imitating, or rather, simulating a MIPS machine running
KUDOS.\\
Unlike emulating a system, the simulator will execute every instruction
exactly
as a hardware machine might do. This allows the developer to not only see how
a program behaves, but also the internal state of the machine, its memory,
registers and so on.\\

