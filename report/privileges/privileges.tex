\label{sec:privileges}
To tighten the security, prevent software faults, and to protect the user from
malware, most modern hardware support mechanisms to detect the malicious behaviour
and act accordingly.\\
Most computer operating systems have access to different resources, ranging from
widely available devices to critical subsystems. To prevent malicious
program or defective code to take over the machine, it is desireable to keep
track of the execution, limiting the access to certain systems.\\
While limited checks can be done by the operating system, it is much more
secure and efficient to let the hardware, in our case, the CPU, to keep track
of each action, faulting on any suspicious activity.\\
This protection mechanism is often called Protection Rings, where the privilege
of a program is determined by the position on a privilege ring --- the inner ring 0
often called "kernel" for being the most privileged, to the outtermost ring,
often used by user applications\cite{see_mips_run}. Depending on the architecture, the protection
ring can have multiple layers.

\subsection{Kernel and User mode}
MIPS supports 2 privilege levels\footnote{All MIPS processors after R4000 model have a
third \textit{supervisor} mode. This mode is mostly ignored by operating systems,
and will not be discussed or used here.\cite{see_mips_run}}, called the kernel and user mode,
distinguished by a bit in the status register. While some competing processors support
more levels, such as x86 supporting 4\cite{intelmanual} levels, they are very
rarely used in practice.\\
The usage of the two privilege modes is very intuitive from the perspective of
the OS --- kernel mode is used
by the operating system and its kernel, while user mode is used for user
applications only.\\
There are multiple functionalities these modes control, ranging from memory access
to instruction access.

\subsubsection{Memory and IO access}
In user mode, a MIPS32 processor can only access the bottom half of the memory
space, that is, the first 2GB. For addresses over, the processor will check
the top bit of the address and the privilege level, restricting non-kernel mode.
However, if the processor has a memory-management unit (discussed in section \ref{sec:mmu}),
the operating itself can control the process access to the memory, preventing
any reads and writes to restricted areas.

\subsubsection{Kernel Instructions}
Some instructions are restricted to the kernel mode only, because they can have
system-wide implications. These instructions are called privileged
instruction, and consist mainly of instructions that move data to and from
co-processor 0, such as \texttt{mfc0} and \texttt{mtc0}.


\subsection{Implementation}
In the simulator, there is no separate need for a module checking the
privileges of the executing process. The privilege checks are done every time
the processor executes an operation with special need, by inspecting the 3rd
and 4th bit of the status register, called the KSU\footnote{In COD5, the lower
bit of this field is called \texttt{UM} for User-Mode\cite{COD5}}.\\
If the process does not satisfy the privilege check, an appropriate exception
is returned from the originating function. This exception is then handled in the WB
stage, described in section \ref{sec:exceptions}.\\
It is to be noted that when handling exceptions, if the EXL flag is on, the
processor is automatically assumed to be in kernel-mode\cite{see_mips_run}.
