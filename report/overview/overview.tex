This report serves as documentation to the MIPS simulator, codename
\texttt{See KUDOS Run}, written as a part of a bachelor project at University
of Copenhagen. See KUDOS Run is a machine simulator, containing pipelined CPUs,
memory-management unit, IO-mapped devices as well as external devices.\\
The simulator and the version-controlled source-code is located at \url{https://github.com/JanmanX/MIPS-Simulator}.
\subsection*{Building the simulator}
After downloading the source, the simulator can be built using:
\begin{verbatim}
$ make
\end{verbatim}
The simulator should be built in the \texttt{./bin/} directory as \texttt{mips-sim}.\\
The simulator can be tested by running the testing script:\\
\begin{verbatim}
$ ./run_tests.sh
\end{verbatim}
\subsection*{Running the simulator}
The simulator has a variety of command-line options, which can all be listed
using the \texttt{-h} or \texttt{--help} flag.\\
Running the simulator can be as simple as only giving the program as the first
command-line argument:
\begin{verbatim}
$ ./bin/mips-sim tests/test_add_addiu.elf
$ echo $?
60
\end{verbatim}
To run a program with debugging enabled, the \texttt{-d} flag is supplied, which
will let the user step the instructions:
\begin{verbatim}
$ ./bin/mips-sim -d tests/test_add_addiu.elf
PC: 0x80010000
0x2408000A	ADDIU  rs = zero = 0x00000000,  rt = t0 = 0x00000000,  imm = 0x0000000A
> q
\end{verbatim}
