Although the primary objective of the simulator is not the performance, it is
an interesting aspect to study.\\
The pipeline approach of the simulator takes a lot of processing to perform,
and is implemented only for academic purposes --- to study the state and behaviour
of the processor. It goes without saying that the pipeline is not needed to
emulate the target CPU, as the emulated instructions are not held back by a time
restricting clock, nor do they have to wait for result from the previous instruction to
continue.\\

\subsection{Comparison}
The simulator is compared to its predecessor YAMS using GNU debugger (GDB). GNU debugger
is a program used to debug, control and inspect other programs. It is a widely
used and very powerful tool used for development on Linux\cite{gdb}.\\
In GDB, a breakpoint is set on every simulator clock tick, that is, every time
the simulator reaches to the start cycle of simulating a new instruction. Then,
GDB steps to next instructions, until the breakpoint is reached again. This
process is repeated a number of times, the number of "steps" saved and printed
at the end. A GDB script \texttt{perf.gdb} is used for this task, which consists
of two \texttt{while} loops - one iterating over the number of simulated instructions,
and the other one counting the host instructions used. At the end of the script,
GDB prints \texttt{\$i}, which is the number of simulated instructions executed,
and \texttt{\$count}, which is the number of host instructions used. The script
can be seen on figure \ref{fig:gdb_perf}.
\begin{figure}
\lstinputlisting[caption=]{../perf.gdb}
\caption{GDB script to test performance.}
\label{fig:gdb_perf}
\end{figure}

Due to compatibility, the simulator will executing KUDOS, while YAMS will be
executing BUENOS. Although this might cause the simulators to execute different
instructions, the test runs enough instructions to approximate their performance.

\begin{center}
    \begin{tabular}{ | c | c | c | c | r | }
    \hline
    & i = 10 & i = 1000 & ratio \\ \hline \hline
    \texttt{YAMS} & 3545 & 430328 &  429\\ \hline
    \texttt{Simulator} & 9218 & 788851 & 790 \\ \hline
    \end{tabular}
\end{center}

%
%CKUDOS RUn
%\$1 = 92183
%\$2 = 10
%
%\$1 = 788851
%\$2 = 1000
%
%
%YAMS
%\$1 = 3545
%\$2 = 10
%
%\$3 = 430328
%\$4 = 1000

By this very primitive test, YAMS is approximatelly 2 times faster, as it uses
the half the instructions to emulate an instruction.\\
On the testing computer, an Intel® Core™ i5-6200U Processor is used, with turbo
frequency up to 2.80GHz\cite{intel:i5}. If fully utilized, the simulator can on
this machine simulate a very fast MIPS processor with over 3.5MHz clock rate,
which is more than enough for the KUDOS operating system:
$$\frac{2.80\mathit{ GHz}}{790} \approx 3.5443\mathit{ MHz}$$
As the YAMS simulator runs at default 1 MHz\cite{yams}, the speed of 3.5 MHz yields enough
room for improvements and additional systems to be implemented in the simulator.
