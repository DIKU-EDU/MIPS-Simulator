MIPS sees Input/Output (IO) devices as a set of special-purpose registers. These
special registers are the processors only way of communicating with a given
device.\\
For every device, there 3 types of registers\cite{cs_uwm:memory_mapped_io}:
\begin{itemize}
	\item \texttt{Status registers}\\
	Provide information about the underlying device. These registers are
	read-only for the CPU.

	\item \texttt{Control registers}\\
	Used to communicate and control the device. These registers are
	writeable for the CPU, but may not always be readable.

	\item \texttt{Data registers}\\
	Used for the actual data-transport. For example, the latest key pressed
	on the keyboard might be stored in this register.
\end{itemize}

Depending on the type of the IO device, the device can be represented from a
few registers to a dozen. For example, a mouse or a keyboard only transmits few
bytes of information at a time, needing only a few registers, while graphic
adaptors or disk drives might need more\cite{cs_uwm:memory_mapped_io}.\\
These special-purpose registers are located in the RAM, mapped to a certain
segment. A device controller maintains a list of these registers, and maps new
devices to the memory.\cite{britton:mips}\\
On MIPS32, the highest 64 kilobytes (\texttt{0xffff0000 - 0xffffffff}) in the
available memory are used for mapping these special registers\cite{cs_uwm:memory_mapped_io}.
The memory mapping the IO area must not be cached, as it can cause major problems
for the caches, and heavy delays due to many cache-misses\cite{see_mips_run}.
