\label{sec:exceptions}
As mentioned in section \ref{sec:cpu_architecture_interrupts}, exceptions and
interrupts are a way to signal the CPU of an event, that needs immediate
attention.\\
In MIPS, everything that defers the usual execution path of a program to attend to
another event, such as interrupts, traps, syscall, are called exceptions\cite{see_mips_run}.
Exceptions are all handled the same way in MIPS.

\subsection{Precise Exceptions}
Due to the pipelined nature of MIPS (and other RISC architectures), exceptions
can occur anywhere in the pipeline. That is, when an exception is raised, there
are probably uncommitted instructions\footnote{Instructions, whose result is not
yet written back to the registers.} in the pipeline. For simplicity of both hardware and
software, it is desireable to handle an exception exactly when it occurs. In
MIPS, precise exceptions means that all instructions preceding the exception
victim\footnote{The instruction that caused the exception.}, are committed, while
all instructions after are not.\\
Precise exceptions give the programmer the usual, sequential view of the
program. However, precise exceptions in MIPS can be very expensive, potentially
clearing the whole pipeline of otherwise faultless instructions.

\subsection{Handling Exceptions}
Depending on the type of the exception, a different exception handler must be
used.\\
Many other architectures, such as the x84 and later x86\_64 use vectored
interrupts. In vectored interrupts, a table of addresses are kept in the
Interrupt Vector Table (IVT), specifying start-addresses of different
interrupt handlers. When an exception is raised, the type of exception is
determined by some microcode or other hardware, and an appropriate handler is
chosen from the IVT\footnote{A x86 assembly
programmer might note that \texttt{int 0x80} in Linux calls the interrupt handler,
located at index 128 in the IVT.}\cite{osdev:interrupt_vector_table}\cite{see_mips_run}. \\
Although this approach is arguably faster, it is not used efficiently in
practice. In many modern operating systems, all interrupt handlers have a lot
of code in common. To avoid redundant code and large interrupt handlers, the
handlers often jump to a shared location in which the redundant code is located,
for example, to save the current registers on the stack.
First, by jumping to a common location, the purpose of the IVT is (almost) defeated,
as the time saved is being spent by jumping to another location anyhow. Secondly, the
information about the interrupt retreived from pure hardware or microcode is
very limited without building very complex hardware. A large OS would have to
analyse the exception further.\\\\
Although MIPS32 has a type of an interrupt vector table\cite{see_mips_run}, it is very rarely used due to its
flaws mentioned above, and the fact that each interrupt handler is only 32
instructions long - something no real exception handler can fit into\cite{see_mips_run}.\\
Due to the speeds of modern processors, MIPS32 exception handling routines always
start at the same address, when an exception arises. \\
This exception handler is by default located in the
unmapped and uncached memory segment KSEG1, address \texttt{0x80000080}, although
this can be changed by modifying the CP0 \texttt{EBase} register, while also
setting the status register bit \texttt{BEV}\cite{see_mips_run}.

\subsection{Interrupts}
\label{sec:exceptions_interrupts}
Hardware caused exceptions are often called "interrupts" in the MIPS world.
While these interrupts are handled just as any other type of exception, they
are not triggered in the code by the means of a syscall instruction. Since
interrupts happen from outside the CPU, a system has to be implemented to notify
the processor of the external event.\\
Like in many other processor architectures, two pins are added to the MIPS CPU,
which are used to signal of a hardware interrupt occurance\cite{cs_pitt:exceptions}.\\
The first pin is called the Interrupt Request (IRQ), which indicates a
pending interrupt. The second pin is the Interrupt Acknowledge (IACK), and is
controlled by the processor to indicate that an interrupt request is acknowledged
and is being serviced. The IACK pin is needed, since the CPU might be busy servicing
another exception, not being able to attend the hardware interrupt immediatelly.
Thus, the IRQ pin can be enabled over multiple clock-cycles before it is
acknowledged\cite{cs_pitt:exceptions}. While it may sound that having a hardware
device wait several clock-cycles results in delayed output, it should be noted
how fast MIPS CPUs (and many others) are, and how many instructions they are
able to carry out before the external device even notices.\\
It should be noted that this is another point where MIPS32 deviates from its
more mainstream x86 counterpart. While the MIPS processor knows that an hardware
interrupt has occured, it does not know which device raised it. To detect the
device that raised the interrupt, it must scan all the device IO registers,
discussed in section \ref{sec:io}. This approach is called "Polled Interrupts".

\subsubsection{Polled Interrupts and Vectored Interrupts}
In polled interrupts, the processor does not know the hardware source of the
interrupt, and must therefore check the status IO register of each device in
order to determine if the device needs to be serviced\cite{cs_pitt:exceptions}.
This method is used in MIPS, for simplicity of the hardware, and because exceptions
are handled by generic exception handler code as well.\\
Another method of handling interrupts is using vectored interrupts, used by
the x86 architecture. In vectored interrupts, much like vectored exceptions, the
processor jumps to a certain interrupt handling code, based on the interrupt type.
However, since the event is external to the processor (caused by hardware),
additional wires must be attached to the CPU, informing it of the IRQ number,
which is unique for each type of device\cite{osdev:interrupts}. This requires
a lot of added hardware complexity in the form of an interrupt controller,
such as the well known Intel 8259 IRQ controller, which can queue up
multiple interrupts, even prioritizing them according to the need of the CPU\cite{osdev:interrupts}.

\subsubsection{Exception procedure}
\label{sec:exception_procedure}
Before exception handler code is executed, the processor must perform certain
actions, required to correctly recover from the exception. These steps are\cite{see_mips_run}:
\begin{enumerate}
\item Setup the \texttt{EPC} register, which hold the address of the victim
instruction. This is used when returning from the exception, so that the
execution can continue from the correct position. \texttt{EPC} is very similar to the
general-purpose \texttt{\$RA}, which holds the return-address for user
programs.
\item Flip on the \texttt{EXL} bit in the status register. This bit forces the
CPU into kernel-mode, giving it higher privileges.
\item Set the \texttt{Cause} register, which holds all the information about
the exception.
\item Jump to \texttt{0x80000080}, where the exception handler code starts.
\end{enumerate}
When the exception handler code is finished, the \texttt{eret} instruction should
be used. Similar to the \texttt{ret} instruction, \texttt{eret} fetches the return
address from \texttt{EPC} register, clears the \texttt{EXL} bit in the status
register, and jumps to the return address.


\subsection{Implementation}
In the simulator, an enum structure \texttt{exception\_t} has been implemented, which
holds all the necessary excetion types (none-exception included).
All functions in the simulator, that can cause an exception, such as reading
from the memory, now return an exception to signal the simulator of the event.
In the simulator, this exception is stored in the corresponding pipeline-register,
forwarded to the subsequent stages. When an instruction with an exception code other
than "exception none" reaches the Write-Back stage, the exception handling
mechanism is called.\\
Note that the exceptions are not necessarily handled in the order they are
raised, but rather, the order of the instructions. For example, in the following
assembly code, a memory exception might occur in MEM at the same time as the arithmetic
overflow exception in EX:
\begin{lstlisting}
	...
	lw $t0, 10($t1)
	add $t3, $t2, 100
	...
\end{lstlisting}
However, due to the way exceptions are forwarded in the pipeline, the memory
exception reaches WB stage first, and gets handled first.

\subsubsection{Interrupts}
As mentioned in the interrupt section \ref{sec:exceptions_interrupts}, interrupts
need another way to signal the processor of an exception that needs attention.
In the simulator, for every clock-cycle, we iterate over every external device
that can trigger an interrupt.
In the loop, we check for its interrupt status, and if needed, handle the exception.\\
In our simulator, the interrupt request is checked every clock-cycle. However,
in some simulators, such as SPIM, a short artificial delay is added to enhance
the realism of the simulator\cite{cod5:appendix}.


\subsubsection{Preparing for exception handling}
When an instruction raises an exception, \texttt{handle\_exception(...)} is
called. This function follows the exception entrance procedure very closely,
described in section \ref{sec:exception_procedure}.
First, it clears all previous pipeline stages, as they now are discarded. The
WB stage is kept, as we need to extract some information about the state.
Then, the cause is determined by analyzing the exception, and stored in the
\texttt{CAUSE} register.
Privileged mode (kernel mode) is entered and PC is set to the exception-
handler address to force the jump.
In the end, stage WB is also cleared to ensure no faulty results are written
back.
\iffalse
\begin{lstlisting}[language=C]
bzero(&IF_ID, sizeof(struct reg_if_id));
bzero(&ID_EX, sizeof(struct reg_id_ex));
bzero(&EX_MEM, sizeof(struct reg_ex_mem));
/* Data in MEM_WB is still needed */

/* 1. Save EPC */
core->cp0.regs[REG_EPC] = MEM_WB.next_pc - 4;

uint32_t cause = get_cause(MEM_WB.exception, MEM_WB.is_branch_delay);
core->cp0.regs[REG_CAUSE] = cause;

if(MEM_WB.exception == EXC_AddressErrorLoad
   || MEM_WB.exception == EXC_AddressErrorLoad) {
	core->cp0.regs[REG_BADVADDR] = MEM_WB.BadVAddr;
}

core->cp0.regs[REG_SR] |= SR_EXL;

core->regs[REG_PC] = (uint32_t)0x80000080;

bzero(&MEM_WB, sizeof(struct reg_mem_wb));
\end{lstlisting}
\fi
It is to be noted that this function is \emph{not} pipelined in the simulator,
giving the impression that all this happens in only one clock cycle.\\
Given that this procedure is executed regardless the type of the exception, one
might argue that this might be implemented in microcode or hardware, really only
using 1 clock-cycle.

