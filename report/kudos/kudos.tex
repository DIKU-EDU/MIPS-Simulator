The simulators main target is to run KUDOS, and so, operating systems such as KUDOS
can be run just as any other userspace program.\\
For initial testing purposes, KUDOS is modified to startup, initialize only
the shutdown device, and trigger the shutdown device.\\
When running the simulator (git commit version \texttt{0ff2a1c8}) on the modified
KUDOS, a crash occurs:
\begin{verbatim}
$ bin/mips-sim -l out.txt kudos-mips32
...
[ERROR] src/mem.c, translate_paddr():216:
ADDRESS OVERFLOW, PADDR: 0x40000004.
Out of bounds: 0x38000000
Segmentation fault (core dumped)
\end{verbatim}
Inspecting the code, the crash occurs at \texttt{0x80011034}\footnote{The instructions
are logged when they enter the pipeline, but the instruction causing the fault is
in memory stage.}. Since KUDOS correctly only uses the user segment (\texttt{kuseg})
for user-space programs, the modified version of KUDOS, which does not start
any userspace programs, should \textit{never} write to the middle of the user segment.\\
The faulting instruction address is located just at the start of \texttt{semaphore\_P},
which are not used in the modified KUDOS. Indeed, the simulator has gone wild and
continued execution, where it should clearly have been shutdown, until a crash
occurs.\\
Upon further inspection of the code, a very peculiar behaviour is uncovered ---
the simulator enters a kind of "NOP" sled, that is, a long section of a program
that only consists of instructions that do nothing. This occurs after instruction
located at \texttt{0x80000084}:
\begin{verbatim}
0x8001891C: 0x1466FFFA  BNE  rs = ...
0x80000088: 0x00000000  SLL  rs = ...
0x8000008C: 0x00000000  SLL  rs = ...
0x80000090: 0x00000000  SLL  rs = ...
0x80000094: 0x00000000  SLL  rs = ...
0x80000098: 0x00000000  SLL  rs = ...
0x8000009C: 0x00000000  SLL  rs = ...
\end{verbatim}
Another peculiar behaviour is exposed --- the program executes in kernel segment
(\texttt{> 0x80000000}), but it is not after the KUDOS start-point \texttt{0x80001000}.
However, the instruction prior can be tracked down to be the comparison of the
device typecode of the shutdown function:
\begin{verbatim}
8001891c: bne v1,a2,80018908 <shutdown+0x34>
\end{verbatim}

It is clear that the execution can not continue at the "NOP-sled" addresses,
and therefore, a bug is present in the address translation mechanism in the simulator,
more specifically, the \texttt{translate\_paddr(...)} function of the memory module.
