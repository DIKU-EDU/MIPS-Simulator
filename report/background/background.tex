KUDOS is heavily based on the BUENOS operating system, originally developed at
Aalto University, Finland\cite{readthedocs:kudos}.\\
Although KUDOS has been extended for the Intel IA-32 architecture, it mainly supports
MIPS32 architecture, just as its predecessor. The OS has been developed along
side a machine simulator YAMS (\textit{Yet Another Machine Simulator}), which
is used to run the operating system.\cite{readthedocs:kudos}


\subsection{KUDOS and YAMS}
BUENOS, and thus KUDOS, has been developed very closely with the YAMS simulator.
Since the intetion of the OS is to be used at operating system courses, it is
heavily dependant on the simulator, its interfaces, and its devices.\\
Despite being a very simple and basic simulation of a machine, it is still a
very realistic hardware interface.


\subsection{Analysis}
The KUDOS operating system is written to fully utilize the many of MIPS32 features,
which the students can research and extend.\\
To be able to run KUDOS correctly, the simulator must at least support the features
listed below:
\begin{itemize}
	\item \texttt{Translation Lookaside Buffer}\\
Most noteably, KUDOS lets the students research and extend the Translation
Lookaside Buffer (TLB), which is a cache used to translate virtual addresses\cite{COD5}.
It is an essential element in modern operating systems, and thus, very
interesting to support in the simulator.

	\item \texttt{TTY device}\\
A typewriter, more commonly known as a TTY is required, to be able to communicate
with the operating system. These devices are usually very basic, only being able
to transmit textual messages.

	\item \texttt{Shutdown device}\\
The operating system needs a way to cleanly shutdown. The simulator will setup
this device, and when required, will efficiently bring the system down.

	\item \texttt{Mapping of all memory segments}\\
KUDOS may be extended to fully utilize the entire main memory. As memory is
split into segments in MIPS32, it is essential for the simulator to be able to
handle this.

	\item \texttt{Bootstrapping mechanism}\\
KUDOS is shipped without a bootloader, relying on the simulator to correctly
load the OS and start it from an appropriate point.
\end{itemize}



\subsection{Design Goals}
For the simulator to work flawlessly with KUDOS, it will be heavily inspired by
the YAMS simulator. Since the OS is relying on the simulator, and vice versa,
we do not want to make any drastic changes to the simulator, as to avoid rewriting
parts of KUDOS. The main design goals of the simulator are:\\
\begin{itemize}
\item \textit{Avoid breaking compatibility}\\
For better integration of the simulator into the operating systems course, it is
undesireable to make such big design changes, that would need to be patched in
the OS as well. Even minor changes in KUDOS could break previously working
code, such as assignments and exercises.

\item \textit{Simplicity over performance}\\
The goal of the simulator is to seamlessly interface with the OS, while still
being simple to understand. KUDOS is a very simple OS, and YAMS is
by no means computationally intensive -- the host machine should have no problems
running both. Therefore, we want to focus on the readability rather than on the
performance.

\item \textit{Usability}\\
The simulator should be easy to use, stable, and flexible in regards of its
functionality.
Apart from running the OS and other simulated programs, the users should be able
to stop the running program, to inspect the state of the simulated machine.
\end{itemize}
