The simulator has been written using modules resembling the actual design of
MIPS32 processors. This yields a natural flow of instructions, as well as a very natural
behaviour of the actual code.\\
While the simulator can successfully run all the test-cases supplied along side
the code, as well as handling IO devices, exceptions and interrupts, we have
shown that the simulator still contains bugs in the exception-handling mechanism,
that, when running a skeleton version KUDOS, will cause the simulator to
continue execution at unknown memory addresses, eventually reaching some actual
code, that will make the simulator enter an invalid state, and crash. \\
Due to this unresolved bug, the Translation Lookaside Buffer has not been implemented,
which is a vital subsystem to run the full version of KUDOS.\\
Despite this bug, enough material has been presented
to show that a pipelined MIPS32 simulator with a working MMU, memory-mapped IO,
exception mechanism as well as debugging functions, can be written, with excellent
performance. After comparison to the more field-tested YAMS, the simulator was
only about half as fast, with all the same features as its counterpart. However,
due to the low performance requirements of the KUDOS system, the simulator should
be more than fast enough.


