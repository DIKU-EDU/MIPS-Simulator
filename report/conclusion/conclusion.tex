The simulator has been written using modules resembling the actual design of
MIPS32 processors. This yields a natural flow of instructions, as well as a very natural
behaviour of the actual code.\\
While the simulator can successfully run all the test-cases supplied along side
the code, we have shown that the simulator still contains bugs that, when running
KUDOS, will cause the simulator to continue execution at unknown memory addresses,
eventually reaching some actual code that will make the simulator enter an invalid
state, and crashing. However, despite this bug, enough material has been presented
to show that a pipelined MIPS32 simulator with a working MMU, memory-mapped IO,
exception mechanism as well as debugging functions, can be written, with excellent
performance. After comparison to the more field-tested YAMS, the simulator was
only about half as fast, with all the same features as its counterpart. However,
due to the low performance requirements of the KUDOS system, the simulator should
be more than fast enough.


