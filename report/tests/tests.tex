It is essential to repeatedly perform tests on the simulator, to ensure its
correct behaviour, and in cases, detect and fix errors and other bugs.\\
To very first tests ensure the basic functionality of the simulator, usually
testing each instruction for the correct behaviour. When the simulator gets more
advanced, additional features are added, that also need to be tested.

\subsection{Basic tests}
Tests for the correct behaviour of an instruction is very simple. Each test uses
the related instruction in as many ways as possible, using as few other
instructions as possible. For example, the \texttt{add} instruction is tested
for both being able to add, but also subtract in one test.\\
For simplicity, the tests return their value in \texttt{\$v1}, which the
simulator uses as its exit value.\\
All test programs lie in the \texttt{tests/} subdirectory, with a corresponding
file ending with \texttt{".text"}, containing the expected result.
To run all tests, a bash script \texttt{run\_tests.sh} is stored on the project
root, which runs each test program with the simulator, comparing the return
value with the expected return value.

\subsection{Advanced tests}
Executing instructions is only a part of the simulation. The simulator also
needs to be able to handle certain situations and problems correctly, and
according to the specification.\\
One of these advanced tests is the pipeline testing, which are programs
with instructions structured in a way so that the pipeline forwarding unit and
hazard unit are used repeatedly. If one of these units fail, the result will be
seen immediately on the output value.\\
Other more advanced tests ensure many side-effects of instructions, such as the
branch-delay slot, or correct flag-toggling of the status register on exception
occurrence.

\subsection{Partial conclusion}
Having a suite of testing programs for each instruction was a tremendous help to
find and fix bugs.\\
However, a big issue with the test suite is that some of the tests depend on
other instructions to work, as to setup the test. This is not necessarily
always the case, and can cause false-negatives, or worse, false-positives.
Optimally, the simulator should be able to start in a certain state, defined by
some configuration file, describing the values and flags of all the processor
registers, flags, even pipeline registers. This would help to test certain
cases in the simulator, avoiding possible side-effects of other possibly faulty
instructions.\\
Another small, but useful improvement would be having dynamic tests, that would
generate test values on every run, so that each time a test is run, it uses new
values, covering a bigger area of the instruction, and further increasing the
chance of finding a bug in the simulator.
