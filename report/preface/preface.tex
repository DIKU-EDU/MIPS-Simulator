\subsection*{Updated Problem Statement}
TO BE WRITTEN.

\subsection*{Target Audience and Prerequisites}
This report is intended for anyone interested in computer architecuture, with
the desire to expand the knowledge on the inner workings of MIPS32 processors.\\
The reader is expected to have either recently have taken a computer architecture course
or read \textit{Computer Organization And Design: The Hardware/Software Interface, 5th
edition, by David A. Petterson and John L. Hennessy}, particularly focusing on
the instruction pipelining within the processing unit.\\
Additionally, some fundamental understanding of computer hardware and related lingo is required,
as well as a basic understanding of programming languages and data-structures.


